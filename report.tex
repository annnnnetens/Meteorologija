\documentclass[12pt,a4paper]{article}
\usepackage[utf8]{inputenc}
\usepackage{amssymb}
\usepackage{amsmath}
\usepackage{graphicx}
\usepackage{caption}
\usepackage{listings}
\captionsetup[figure]{labelfont={bf},name={Fig.},labelsep=period}
\captionsetup[table]{labelfont={bf},name={Tab.},labelsep=period}
\usepackage{multicol}


\title{Intervijas uzdevumi LVĢMC datu analītiķa pozīcijai}
% \title{Intervijas uzdevumi LV\c{G}MC datu anal\={\i}ti\c{k}a poz\={\i}cijai}
\author{Anete Valnere}
\date{\today}
\begin{document}
% \maketitle
% \big{Anete Valnere}
\section*{Pirmais uzdevums}


\begin{figure}[h]
    \centering
    \includegraphics[width=7cm]{kastes_kluda.png}
    \caption{Kastu grafiki pa stacijām}
\end{figure}
Pirmajā figūrā redzami kastu grafiki stacijās \(x_1\), \(x_2\) un \(x_3\). Uzreiz ir skaidrs, ka trešajā stacijā daži mērījumi nav pareizi, jo dati atšķiras par vairākām magnitūdām no abām pārējām stacijām.% Lai par to pārliecinātos, izvadīsim visus trešās stacijas datus, kas ir lielāki par pirmās un otrās stacijas maksimālajiem mērījumiem.
% \begin{verbatim}
%            Datums.un.laiks   x1   x2          x3
% 263858 2000-07-09 11:01:00  0.8  2.2  278.969205
% 263859 2000-07-09 11:02:00  1.0  2.4  270.919402
% 263860 2000-07-09 11:03:00  0.9  2.6  245.731204
% 263861 2000-07-09 11:04:00  0.6  2.1  123.176774
% 263862 2000-07-09 11:05:00  0.3  1.3  138.064805
% 263863 2000-07-09 11:06:00  0.0  0.9  112.456679
% 263864 2000-07-09 11:07:00  0.0  1.2  127.830121
% 263865 2000-07-09 11:08:00  0.0  1.7  236.335530
% 263866 2000-07-09 11:09:00  0.0  2.5   47.490301
% 263867 2000-07-09 11:10:00  0.0  3.0  178.308171
% 263868 2000-07-09 11:11:00  0.0  2.6  168.389565
% 263869 2000-07-09 11:12:00  0.0  2.2  309.938427
% \end{verbatim}


Apskatot datus trešajā stacijā, kuri ir lielāki par maksimālajām vērtībām pirmajā un otrajā stacijā, varam ievērot, ka visi šie mērījumi ir veikti devītajā jūlijā secīgās minūtēs, kā arī tie neseko skaidram rakstam, līdz ar to izteiksim pieņēmumu, ka vainīga ir sensora kļūda.

Pēc šo datu pārvēršanas par \texttt{NA}, varam vēlreiz apskatīt kastu grafikus.
\begin{figure}[h]
    \centering
    \includegraphics[width=7cm]{kastes_labs.png}
    \caption{Kastu grafiki pa stacijām}
\end{figure}
Tagad staciju mērījumi ir daudz salīdzināmāki un varam aprēķināt vidējās vērtības.
\begin{table}
    \caption{Vidējās vērtības katrā stacijā pa mēnešiem}
    \centering
    \begin{tabular}{l|c c c}
    Mēnesis & \(x_1\) & \(x_2\) & \(x_3\) \\
        \hline
        1  &  2.212  &  1.984  &  2.690 \\
        2  &  3.184  &  nan  &  3.841 \\
        3  &  4.168  &  2.335  &  2.389 \\
        4  &  3.019  &  2.703  &  2.764 \\
        5  &  1.962  &  2.080  &  2.129 \\
        6  &  2.251  &  2.147  &  2.380 \\
        7  &  2.103  &  2.025  &  2.303 \\
        8  &  2.051  &  1.944  &  2.364 \\
        9  &  1.847  &  1.821  &  2.134 \\
        10  &  2.831  &  2.881  &  3.002 \\
        11  &  3.087  &  3.111  &  3.396 \\
        12  &  2.944  &  2.974  &  3.585
    \end{tabular}
\end{table}

Lai arī dati katrā stacijā noteikti nav neatkarīgi viens no otra (dabas parādību novērojuma iespējamība noteikti ir atkarīga no iepriekšējās minūtes), datu ir tik daudz, ka vienkāršības labad varam pieņemt neatkarību.

Piemeklējot datu sadalījumu, redzam, ka dati pieņem reālas pozitīvas vērtības, un katrā stacijā gandrīz puse no mērījumiem ir starp 0 un 2, kā arī \(75\%\) mērījumu ir zem 4. Līdz ar to sadalījuma grafiks atrodas augstu starp 0 un 4, un no 500 000 mērījumiem trīs stacijās neviens nepārsniedz 16 (protams, izņemot dažus kļūdainos mērījumus). Vispārīgākais sadalījums šādam gadījumam būtu Gamma sadalījums, kuram mums jāatrod forma un mērogs. Diemžēl pat vienkāršam \(\Gamma(k, 1)\) sadalījumam nevar precīzi atrast aptuveno vērtību (estimator) formai \(k\), līdz ar to nāksies pieņemt formu \(k=3,5\), kurai lielākā daļa atrodas zem 4, un tad piemeklēt mērogu \(\theta=0,8\), kas augšējo asti izstiepj pareiza garuma.

TODO: pievienot boxplotu gammai

\section*{Otrais uzdevums}
Failu ielādi nācās veikt pilnīgi atšķirīgos veidos.

TODO: jāsalauž rindiņas
\begin{verbatim*}
Vol_URL = 'https://wis.wmo.int/operational-info/VolumeC1/VolC1.txt'
ESWI_URL = 'https://wis.wmo.int/operational-info/GTS_routeing/ESWI/ESWIroca.txt'

Vol = pd.read_csv(Vol_URL, quoting=csv.QUOTE_ALL, quotechar='"', encoding='latin-1')

if os.path.isfile('ESWIroca.txt'):
    ESWI = pd.read_csv('ESWIroca.txt')
else:
    f = open(urlretrieve(ESWI_URL)[0])
    ESWI = pd.DataFrame(columns=['TTAAii', 'CCCC', 'Receivers'])
    f.readline()    # pirmā rinda nav vajadzīga, tādēļ to jānolasa.
    for line in f:
        objs = line[:-1].split(',')                     # sadala visu pa komatiem, noņem \n
        objs = [item.strip('\"') for item in objs]      # noņem pēdiņas
        sender = objs[0].split()                        # sadala pirmo vārdu divos
        ESWI = pd.concat([ESWI, pd.DataFrame([[sender[0], sender[1], objs[1:]]], columns=ESWI.columns)], ignore_index=True)
    ESWI.to_csv('ESWIroca.txt', index=False)            # saglabā, lai nebūtu tas vēlreiz jādara
\end{verbatim*}
Pirmo failu \texttt{VolC1.txt} var ielādēt ļoti vienkārši, izmantojot \texttt{pandas} bibliotēku, jo tas satur tikai tabulu. Otrajā failā katrā rindiņā ir nezināms skaits elementu, līdz ar to šo failu nevar tiešā veidā pārveidot kā tabulu, taču zinām, ka pirmais elements satur \texttt{TTAAii} un \texttt{CCCC}, savukārt visi atlikušie elementi rindiņā ir ziņas saņēmēji. Sadalot šīs rindiņas attiecīgi un noņemot pēdiņas, mums sanāk jauna tabula \texttt{ESWI} ar trīs kolonnām, kur trešajā atrodas saraksts (list) ar saņēmēju \texttt{CCCC} identifikatoriem.

% \begin{multicols}{2}

% \end{multicols}
TODO: pievienot telegrammu skaitu un valstis

Pielikumā~\ref{ap:II 3} pievienoti failu uzskaitījumi ar prasītajiem \texttt{CCCC} kodiem.

\section*{Trešais uzdevums}




\begin{figure}[ht!]
    \centering
    \includegraphics[width=13cm]{baltija.png}
    \caption{SSR Baltijā 2022-07-06}
    \label{baltija}
\end{figure}
\begin{figure}[ht!]
    \centering
    \includegraphics[width=7cm]{pilsetas.png}
    \caption{Stundas vidējās radiācijas vērtības Latvijas pilsētās}
    \label{pilsetas}
\end{figure}
Jāievēro, ka \ref{pilsetas}.~figūrā pusnakts atrodas nakts beigās. Tas liecina par to, ka laiks ir dots Griničas laika zonā, jo jūlijā Latvijā saule riet ap 22:30 un aust ap 05:30, līdz ar to saules radiācija ir ap nulli šajā laika posmā, taču grafikā izskatās, ka saule lec jau divos, līdz ar to pusnakts ir Griničas laika zonā. Attiecīgi arī \texttt{SSR} ir jāskatās trīs stundas agrāk, un \ref{baltija}.~figūrā ir izmantots lokālais, nevis Griničas laiks. Tā kā \texttt{SSR} ir summārā radiācija, tad par references laiku ir izmantota tās pašas dienas pusnakts, un attiecīgi plkst. 01:00 summārā radiācija nav būtiski palielinājusies, salīdzinot ar pusnakti.


\newpage
\appendix
\section{Uzdevums II 3.}\label{ap:II 3}
\subsection*{Faili mapē \texttt{UMRR}}
\begin{lstlisting}[breaklines]
data/6/NORRKOPING/LATVIA/UMRR/LATVIA_ULLV10_UMRR_00.txt
data/6/NORRKOPING/LATVIA/UMRR/LATVIA_FQLV30_UMRR_XXX.txt
data/6/NORRKOPING/LATVIA/UMRR/LATVIA_ISID11_UMRR_03,09,15,21.txt
data/6/NORRKOPING/LATVIA/UMRR/LATVIA_ISCD10_UMRR_MONTHLY.txt
data/6/NORRKOPING/LATVIA/UMRR/LATVIA_SMLV10_UMRR_18.txt
data/6/NORRKOPING/LATVIA/UMRR/LATVIA_FELV40_UMRR_XXX.txt
data/6/NORRKOPING/LATVIA/UMRR/LATVIA_UKLV10_UMRR_00.txt
data/6/NORRKOPING/LATVIA/UMRR/LATVIA_UELV10_UMRR_00.txt
data/6/NORRKOPING/LATVIA/UMRR/LATVIA_ISCD60_UMRR_MONTHLY.txt
data/6/NORRKOPING/LATVIA/UMRR/LATVIA_FPLV30_UMRR_XXX.txt
data/6/NORRKOPING/LATVIA/UMRR/LATVIA_IUKD10_UMRR_00.txt
data/6/NORRKOPING/LATVIA/UMRR/LATVIA_WOLV30_UMRR_AS REQUIRED.txt
data/6/NORRKOPING/LATVIA/UMRR/LATVIA_VMLV40_UMRR_06.txt
data/6/NORRKOPING/LATVIA/UMRR/LATVIA_FPLV40_UMRR_XXX.txt
data/6/NORRKOPING/LATVIA/UMRR/LATVIA_FPLV10_UMRR_XXX.txt
data/6/NORRKOPING/LATVIA/UMRR/LATVIA_ISMD11_UMRR_00,06,12,18.txt
data/6/NORRKOPING/LATVIA/UMRR/LATVIA_USLV10_UMRR_00.txt
data/6/NORRKOPING/LATVIA/UMRR/LATVIA_SMLV10_UMRR_12.txt
data/6/NORRKOPING/LATVIA/UMRR/LATVIA_IUSD10_UMRR_00.txt
data/6/NORRKOPING/LATVIA/UMRR/LATVIA_CULV10_UMRR_MONTHLY.txt
\end{lstlisting}
\subsection*{Faili mapē \texttt{ESWI}}
\begin{lstlisting}[breaklines]
data/6/NORRKOPING/SWEDEN/ESWI/SWEDEN_WSSN31_ESWI_AS REQUIRED.txt
data/6/NORRKOPING/SWEDEN/ESWI/SWEDEN_SMVD01_ESWI_00.txt
data/6/NORRKOPING/SWEDEN/ESWI/SWEDEN_SMVA01_ESWI_18.txt
data/6/NORRKOPING/SWEDEN/ESWI/SWEDEN_IUSD15_ESWI_00,18.txt
data/6/NORRKOPING/SWEDEN/ESWI/SWEDEN_SMVF01_ESWI_18.txt
data/6/NORRKOPING/SWEDEN/ESWI/SWEDEN_IUKN41_ESWI_00,06,12,18.txt
data/6/NORRKOPING/SWEDEN/ESWI/SWEDEN_SMVB01_ESWI_18.txt
data/6/NORRKOPING/SWEDEN/ESWI/SWEDEN_SMVD01_ESWI_06.txt
data/6/NORRKOPING/SWEDEN/ESWI/SWEDEN_SMVD01_ESWI_12.txt
data/6/NORRKOPING/SWEDEN/ESWI/SWEDEN_ISMD61_ESWI_00,06,12,18.txt
data/6/NORRKOPING/SWEDEN/ESWI/SWEDEN_SASN33_ESWI_H+20,H+50.txt
data/6/NORRKOPING/SWEDEN/ESWI/SWEDEN_IUXD13_ESWI_00,12.txt
data/6/NORRKOPING/SWEDEN/ESWI/SWEDEN_IUKD11_ESWI_00.txt
data/6/NORRKOPING/SWEDEN/ESWI/SWEDEN_SIVA21_ESWI_09.txt
data/6/NORRKOPING/SWEDEN/ESWI/SWEDEN_SIVA21_ESWI_21.txt
data/6/NORRKOPING/SWEDEN/ESWI/SWEDEN_IUKD15_ESWI_00,18.txt
data/6/NORRKOPING/SWEDEN/ESWI/SWEDEN_SIVF21_ESWI_09.txt
data/6/NORRKOPING/SWEDEN/ESWI/SWEDEN_SIVF21_ESWI_21.txt
data/6/NORRKOPING/SWEDEN/ESWI/SWEDEN_WVSN31_ESWI_AS REQUIRED.txt
data/6/NORRKOPING/SWEDEN/ESWI/SWEDEN_SIVD21_ESWI_03.txt
data/6/NORRKOPING/SWEDEN/ESWI/SWEDEN_SIVB21_ESWI_21.txt
data/6/NORRKOPING/SWEDEN/ESWI/SWEDEN_SIVB21_ESWI_09.txt
data/6/NORRKOPING/SWEDEN/ESWI/SWEDEN_UASN61_ESWI_AS REQUIRED.txt
data/6/NORRKOPING/SWEDEN/ESWI/SWEDEN_IUKD01_ESWI_00.txt
data/6/NORRKOPING/SWEDEN/ESWI/SWEDEN_SIVD21_ESWI_15.txt
data/6/NORRKOPING/SWEDEN/ESWI/SWEDEN_IUSD16_ESWI_00,12.txt
data/6/NORRKOPING/SWEDEN/ESWI/SWEDEN_SNSN86_ESWI_01, 02, 04, 05, 07, 08, 10, 11, 13, 14, 16, 17, 19, 20 ,22, 23.txt
data/6/NORRKOPING/SWEDEN/ESWI/SWEDEN_ISND61_ESWI_01,02,04,05,07,08,10,11,13,14,16,17,19,20,22,23.txt
data/6/NORRKOPING/SWEDEN/ESWI/SWEDEN_SIVA21_ESWI_03.txt
data/6/NORRKOPING/SWEDEN/ESWI/SWEDEN_SIVA21_ESWI_15.txt
data/6/NORRKOPING/SWEDEN/ESWI/SWEDEN_IUKD06_ESWI_00,12.txt
data/6/NORRKOPING/SWEDEN/ESWI/SWEDEN_SASN31_ESWI_H+20,H+50.txt
data/6/NORRKOPING/SWEDEN/ESWI/SWEDEN_FTSN31_ESWI_05,11,17,23.txt
data/6/NORRKOPING/SWEDEN/ESWI/SWEDEN_SIVB21_ESWI_15.txt
data/6/NORRKOPING/SWEDEN/ESWI/SWEDEN_FCSN31_ESWI_02,05,08,11,14,17,23.txt
data/6/NORRKOPING/SWEDEN/ESWI/SWEDEN_SIVF21_ESWI_15.txt
data/6/NORRKOPING/SWEDEN/ESWI/SWEDEN_ISID61_ESWI_03,09,15,21.txt
data/6/NORRKOPING/SWEDEN/ESWI/SWEDEN_SIVF21_ESWI_03.txt
data/6/NORRKOPING/SWEDEN/ESWI/SWEDEN_SIVD21_ESWI_09.txt
data/6/NORRKOPING/SWEDEN/ESWI/SWEDEN_SIVD21_ESWI_21.txt
data/6/NORRKOPING/SWEDEN/ESWI/SWEDEN_SIVB21_ESWI_03.txt
data/6/NORRKOPING/SWEDEN/ESWI/SWEDEN_WOSN42_ESWI_XXX.txt
data/6/NORRKOPING/SWEDEN/ESWI/SWEDEN_SMVA01_ESWI_06.txt
data/6/NORRKOPING/SWEDEN/ESWI/SWEDEN_SMVA01_ESWI_12.txt
data/6/NORRKOPING/SWEDEN/ESWI/SWEDEN_SMVF01_ESWI_00.txt
data/6/NORRKOPING/SWEDEN/ESWI/SWEDEN_IUSD01_ESWI_00.txt
data/6/NORRKOPING/SWEDEN/ESWI/SWEDEN_SMVB01_ESWI_00.txt
data/6/NORRKOPING/SWEDEN/ESWI/SWEDEN_STSN42_ESWI_XXX.txt
data/6/NORRKOPING/SWEDEN/ESWI/SWEDEN_IUSN41_ESWI_00,06,12,18.txt
data/6/NORRKOPING/SWEDEN/ESWI/SWEDEN_UASN71_ESWI_AS REQUIRED.txt
data/6/NORRKOPING/SWEDEN/ESWI/SWEDEN_IUSD06_ESWI_00,12.txt
data/6/NORRKOPING/SWEDEN/ESWI/SWEDEN_SMVA01_ESWI_00.txt
data/6/NORRKOPING/SWEDEN/ESWI/SWEDEN_ISND22_ESWI_01,02,04,05,07,08,10,11,13,14,16,17,19,20,22,23.txt
data/6/NORRKOPING/SWEDEN/ESWI/SWEDEN_FQSN40_ESWI_XXX.txt
data/6/NORRKOPING/SWEDEN/ESWI/SWEDEN_STSN43_ESWI_XXX.txt
data/6/NORRKOPING/SWEDEN/ESWI/SWEDEN_SMVF01_ESWI_06.txt
data/6/NORRKOPING/SWEDEN/ESWI/SWEDEN_SMVF01_ESWI_12.txt
data/6/NORRKOPING/SWEDEN/ESWI/SWEDEN_ISCD01_ESWI_MONTHLY.txt
data/6/NORRKOPING/SWEDEN/ESWI/SWEDEN_IUSD11_ESWI_00.txt
data/6/NORRKOPING/SWEDEN/ESWI/SWEDEN_SMVB01_ESWI_06.txt
data/6/NORRKOPING/SWEDEN/ESWI/SWEDEN_SMVB01_ESWI_12.txt
data/6/NORRKOPING/SWEDEN/ESWI/SWEDEN_IUKD16_ESWI_00,12.txt
data/6/NORRKOPING/SWEDEN/ESWI/SWEDEN_SMVD01_ESWI_18.txt
\end{lstlisting}
\subsection*{Faili mapē \texttt{EEMH}}
\begin{lstlisting}[breaklines]
data/6/NORRKOPING/ESTONIA/EEMH/ESTONIA_STEO11_EEMH_XXX.txt
data/6/NORRKOPING/ESTONIA/EEMH/ESTONIA_FEEO40_EEMH_XXX.txt
data/6/NORRKOPING/ESTONIA/EEMH/ESTONIA_SIEO40_EEMH_15.txt
data/6/NORRKOPING/ESTONIA/EEMH/ESTONIA_SMEO41_EEMH_06.txt
data/6/NORRKOPING/ESTONIA/EEMH/ESTONIA_SMEO41_EEMH_12.txt
data/6/NORRKOPING/ESTONIA/EEMH/ESTONIA_ISND21_EEMH_01,02,04,05,07,08,10,11,13,14,16,17,19,20,22,23.txt
data/6/NORRKOPING/ESTONIA/EEMH/ESTONIA_SIEO40_EEMH_03.txt
data/6/NORRKOPING/ESTONIA/EEMH/ESTONIA_IUXD11_EEMH_00.txt
data/6/NORRKOPING/ESTONIA/EEMH/ESTONIA_SNEO41_EEMH_01,02,04,05,07,08,10,11,13,14,16,17,19,20,22,23.txt
data/6/NORRKOPING/ESTONIA/EEMH/ESTONIA_SIEO20_EEMH_15.txt
data/6/NORRKOPING/ESTONIA/EEMH/ESTONIA_ISAD30_EEMH_EVERY 10 MINUTES.txt
data/6/NORRKOPING/ESTONIA/EEMH/ESTONIA_STEO10_EEMH_XXX.txt
data/6/NORRKOPING/ESTONIA/EEMH/ESTONIA_SMEO41_EEMH_00.txt
data/6/NORRKOPING/ESTONIA/EEMH/ESTONIA_FQEO30_EEMH_XXX.txt
data/6/NORRKOPING/ESTONIA/EEMH/ESTONIA_SIEO20_EEMH_03.txt
data/6/NORRKOPING/ESTONIA/EEMH/ESTONIA_SNEO40_EEMH_01,02,04,05,07,08,10,11,13,14,16,17,19,20,22,23.txt
data/6/NORRKOPING/ESTONIA/EEMH/ESTONIA_FPEO40_EEMH_XXX.txt
data/6/NORRKOPING/ESTONIA/EEMH/ESTONIA_SMEO10_EEMH_12.txt
data/6/NORRKOPING/ESTONIA/EEMH/ESTONIA_SMEO10_EEMH_06.txt
data/6/NORRKOPING/ESTONIA/EEMH/ESTONIA_USEO10_EEMH_00.txt
data/6/NORRKOPING/ESTONIA/EEMH/ESTONIA_IUSD11_EEMH_00.txt
data/6/NORRKOPING/ESTONIA/EEMH/ESTONIA_SMEO40_EEMH_18.txt
data/6/NORRKOPING/ESTONIA/EEMH/ESTONIA_VMEO40_EEMH_XXX.txt
data/6/NORRKOPING/ESTONIA/EEMH/ESTONIA_ISMD40_EEMH_00,06,12,18.txt
data/6/NORRKOPING/ESTONIA/EEMH/ESTONIA_SMEO10_EEMH_00.txt
data/6/NORRKOPING/ESTONIA/EEMH/ESTONIA_IUKD10_EEMH_00.txt
data/6/NORRKOPING/ESTONIA/EEMH/ESTONIA_SIEO41_EEMH_21.txt
data/6/NORRKOPING/ESTONIA/EEMH/ESTONIA_SIEO41_EEMH_09.txt
data/6/NORRKOPING/ESTONIA/EEMH/ESTONIA_SUEO40_EEMH_XXX.txt
data/6/NORRKOPING/ESTONIA/EEMH/ESTONIA_SMEO10_EEMH_18.txt
data/6/NORRKOPING/ESTONIA/EEMH/ESTONIA_FPEO43_EEMH_XXX.txt
data/6/NORRKOPING/ESTONIA/EEMH/ESTONIA_UKEO10_EEMH_00.txt
data/6/NORRKOPING/ESTONIA/EEMH/ESTONIA_SNEO21_EEMH_01,02,04,05,07,08,10,11,13,14,16,17,19,20,22,23.txt
data/6/NORRKOPING/ESTONIA/EEMH/ESTONIA_ISCD60_EEMH_MONTHLY.txt
data/6/NORRKOPING/ESTONIA/EEMH/ESTONIA_SMEO40_EEMH_00.txt
data/6/NORRKOPING/ESTONIA/EEMH/ESTONIA_ISID20_EEMH_03,09,15,21.txt
data/6/NORRKOPING/ESTONIA/EEMH/ESTONIA_ISXD64_EEMH_DAILY.txt
data/6/NORRKOPING/ESTONIA/EEMH/ESTONIA_ISMD10_EEMH_00,06,12,18.txt
data/6/NORRKOPING/ESTONIA/EEMH/ESTONIA_SIEO41_EEMH_03.txt
data/6/NORRKOPING/ESTONIA/EEMH/ESTONIA_ULEO10_EEMH_00.txt
data/6/NORRKOPING/ESTONIA/EEMH/ESTONIA_SIEO41_EEMH_15.txt
data/6/NORRKOPING/ESTONIA/EEMH/ESTONIA_CSEO10_EEMH_MONTHLY.txt
data/6/NORRKOPING/ESTONIA/EEMH/ESTONIA_SMEO40_EEMH_12.txt
data/6/NORRKOPING/ESTONIA/EEMH/ESTONIA_SMEO40_EEMH_06.txt
data/6/NORRKOPING/ESTONIA/EEMH/ESTONIA_IUKD11_EEMH_00.txt
data/6/NORRKOPING/ESTONIA/EEMH/ESTONIA_SIEO40_EEMH_09.txt
data/6/NORRKOPING/ESTONIA/EEMH/ESTONIA_SIEO40_EEMH_21.txt
data/6/NORRKOPING/ESTONIA/EEMH/ESTONIA_ISCD10_EEMH_MONTHLY.txt
data/6/NORRKOPING/ESTONIA/EEMH/ESTONIA_IUSD10_EEMH_00.txt
data/6/NORRKOPING/ESTONIA/EEMH/ESTONIA_ISND40_EEMH_01,02,04,05,07,08,10,11,13,14,16,17,19,20,22,23.txt
data/6/NORRKOPING/ESTONIA/EEMH/ESTONIA_SMEO41_EEMH_18.txt
data/6/NORRKOPING/ESTONIA/EEMH/ESTONIA_ISID40_EEMH_03,09,15,21.txt
data/6/NORRKOPING/ESTONIA/EEMH/ESTONIA_WOEO30_EEMH_XXX.txt
data/6/NORRKOPING/ESTONIA/EEMH/ESTONIA_SIEO20_EEMH_09.txt
data/6/NORRKOPING/ESTONIA/EEMH/ESTONIA_SIEO20_EEMH_21.txt
data/6/NORRKOPING/ESTONIA/EEMH/ESTONIA_UEEO10_EEMH_00.txt
\end{lstlisting}





\end{document}